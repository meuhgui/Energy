\documentclass{article}
\usepackage[T1]{fontenc}
\usepackage[utf8]{inputenc}
\usepackage[francais]{babel}

%% This package is necessary to use \includegraphics.
\usepackage{graphicx}

%% This package is necessary to define hyperlinks.
\usepackage{hyperref}

%% This package is needed to enchance mathematical formulas.
\usepackage{amsmath}

%% This package is needed to draw figures
\usepackage{pgf,tikz}
\usetikzlibrary{arrows}
\title{Rapport de projet - Interfaces Graphiques}

\author{Rémi Lévy, Guillermo Morón Usón}

\begin{document}
\maketitle
\selectlanguage{french}

Pour lancer le programme, voir le fichier README à la racine du projet.

\section{Fonctionnement}
Au premier lancement du programme le répertoire {\tt \$HOME/.energy/} est créé,
ainsi que les sous-répertoire {\tt editable/} et {\tt playable/}. Ces dossiers
serviront à stocker les niveaux jouables et éditables. Ainsi, pour supprimer
complètement cette implémentation du jeu Energy, il suffit de supprimer le
répertoire {\tt \$HOME/.energy/} et l'archive décompressée contenant le code
source et executable. Pour lancer le jeu, il faut se placer à la racine de
l'archive décompressée et exécuter la commande {\tt ./gradlew run}.

Après l'installation, une fenêtre s'affiche à l'écran vous proposant deux
banques de niveaux, les niveaux {\em jouables} et les niveaux {\em éditables}.
Les niveaux prédéfinis sont jouables mais non éditables, cependant tout nouveau
niveau créé est jouable et éditable. Selon la banque sélectionnée, l'utilisateur
peut tenter de finir le niveau ou éditer entièrement le niveau, les
modifications étant effectives si le circuit est dans une configuration
gagnante.

\section{Packages}

Le {\em package} principal est {\tt energy}, dans lequel se trouve le fichier
{\tt App.java} et les {\em sous-packages} {\tt model}, {\tt view} et
{\tt controller}. Le fichier {\tt App.java} est le point d'entrée du programme,
celui qui construit l'application.

\subsection{Model}

Le {\em package} {\tt model} contient les données de l'application, dont les
objets les plus importants suivants:
\begin{itemize}
\item {\tt Level} : encapsule un {\tt Circuit} et son identifiant.
\item {\tt Circuit} : encapsule une collection de {\tt Tile} connectées entre
  elles.
\item {\tt Tile}: encapsule une forme, une collection de côtés connectés, une
  {\tt Position}, formant collectivement une matrice dans {\tt Circuit}, ainsi
  qu'un {\tt Component}, représentant les éléments posés sur les tuiles tels les
  lampes ou les sources d'énergie.
\item {\tt ReadOnlyCircuit} : représente un {\tt Circuit} accessible uniquement
  en lecture. Est utilisé notamment pour mettre à jour la Vue.
\item {\tt EditableLevel} : définit ce qu'est un niveau éditable. Est utilisé
  notamment en tant que Modèle lorsque l'on édite un niveau. C'est le patron de
  conception Adapteur.
\item {\tt PlayableLevel} : définit ce qu'est un niveau jouable. Est utilisé en
  tant que Modèle lorsque l'on joue à un niveau.
\end{itemize}

La forme des tuiles et leurs composant sont encapsulés dans des énumérations et
sont des attributs des tuiles. Ainsi, on évite la duplication de code dûe aux
multiples combinaisons de forme/composant possibles.

\subsection{View}
Le {\em package} {\tt view} s'occupe de l'affichage des composants sur
l'interface graphique. Les objets les plus importants sont les suivants:
\begin{itemize}
\item {\tt ScreenSwitch} : interface qui permet aux écrans de sélection de
  niveau, edition de niveau et jeu de passer d'un écran à l'autre selon l'état
  de l'application. La fonction {\tt next(int, bool)} permet de montrer à
  l'écran la représentation du niveau d'identifiant donné en mode édition ou jeu
  selon la valeur du drapeau.
\item {\tt CircuitView} : affiche sous forme de plateau rectangulaire les tuiles
  d'un niveau. Les tuiles et la disposition sont récupérées à partir d'un
  {\tt Circuit} accessible en lecture seule, un {\tt ReadOnlyCircuit}, on limite
  ainsi la connaissance des données au strict nécessaire.
\item {\tt TileView} : permet de charger, placer et orienter les images
  constituant la représentation d'une {\tt Tile} à l'écran. Les images ne sont
  chargées qu'au besoin et qu'une fois.
\item {\tt LevelView} : encapsule la représentation graphique d'un niveau,
  ajoutant au besoin les boutons nécessaires pour l'édition de niveau ainsi que
  les boutons de retour à l'écran principal. Observe le Modèle et met à jour
  la représentation du circuit lorsqu'il est notifié d'un changement d'état.
\end{itemize}

\subsection{Controller}
Le {\em package} {\tt controller} contient les classes qui gèrent les entrées
de l'utilisateur et agissent sur le modèle en conséquence. Lorsque l'une des ces
actions se produit, les contrôleurs transmettent au Modèle les données et si son
état change, notifie la Vue. Il y a deux contrôleurs, l'un pour l'édition de
niveau et l'autre pour le jeu. Leur généralisation sert uniquement à déterminer
la tuile sur laquelle l'utilisateur a agit.

\begin{itemize}
\item {\tt LevelController} : un MouseInputAdapter qui permet de déterminer
  la position de la tuile sur laquelle l'utilisateur a cliqué. C'est la classe
  de base pour les contrôleurs de programme elle n'est pas utilisé directement.
\item {\tt EditorController} : c'est le contrôleur à utiliser lorsque
  l'utilisateur sélectionne un niveau à éditer. Il encapsule un
  {\tt EditableLevel} qui n'autorise que des actions de modification du nombre
  de ligne ou colonnes, des composants et des connexions des tuiles dans le
  circuit.
\item {\tt PlayableLevel} : c'est le contrôleur à utiliser lorsque l'utilisateur
  selectionne un niveau pour jouer. Il limite l'action de l'utilisateur sur le
  circuit à la rotation des tuiles en utilisant un {\tt PlayableLevel} comme
  Modèle.
\end{itemize}

\end{document}
